\documentclass[12pt]{article}
\usepackage{amsmath}
\usepackage{amssymb}
\usepackage{graphicx}
\usepackage{hyperref}
\usepackage[latin1]{inputenc}


\renewcommand{\>}{\rangle}
\newcommand{\<}{\langle}
\newcommand{\cL}{\mathcal{L}}
\newcommand{\cH}{\mathcal{H}}
\newcommand{\C}{\mathbb{C}}
\newcommand{\tr}{\mathrm{Tr}}

\setlength{\marginparwidth}{0pt} \setlength{\hoffset}{0cm}
\setlength{\oddsidemargin}{0pt} \setlength{\topmargin}{1cm}
\setlength{\headheight}{0pt} \setlength{\headsep}{0pt}
\addtolength{\textwidth}{3cm} \addtolength{\textheight}{5cm}

\title{Assignment 1 part 2}
\author{Tyler Duncan}
\date{February 21, 2019}

\begin{document}
\maketitle

\medskip
\noindent
For $A,B\in\C^{d\times d}$, define
\[
\< A | B\>_{\mathrm{Tr}} = \mathrm{Tr}(A^\dagger B)\,.
\]
Prove that the above map defines an inner product on the vector space $\C^{d\times d}$. (In the literature, this inner product is called the trace inner product or Hilbert-Schmidt inner product.)

\medskip
\medskip
\medskip
\noindent
We will use proof by definition. 
\noindent
Assuming the above map to be true, it must have the following attributes:

\medskip
\medskip
\medskip
\noindent

1. It must have linearity in the first argument

2. Conjugate-commutativity

3. Non-negativity

\medskip
\medskip
\medskip
\noindent
We can see that $\<.,.\>$ is linear in the first argument since for every $a,b\in\C$ 
and $A,B,C\in\C^{d\times d}$

\begin{equation}
\begin{aligned}
\<aA + bB, C\> &= \mathrm{Tr}((aA + bB)C^*) \\ 
& = \mathrm{Tr}(aAC^* + bBC^*) \\ 
& = a\mathrm{Tr}(AC^*) + b\mathrm{Tr}(BC^*) \\ 
& = a\<A, C\> + b\<B, C\>
\end{aligned}
\end{equation}

\medskip
\medskip
\medskip
\noindent
We can see $\<.,.\>$ has conjugate-commutativity by 

\begin{equation}
\begin{aligned}
\<A,B\> = \mathrm{Tr}(AB^*) = \mathrm{Tr}((BA^*)^*) = \overline{\mathrm{Tr}(AB^*)} = \overline{\<B,A\>}\\ 
\end{aligned}
\end{equation}

\medskip
\medskip
\medskip
\noindent
Lastly, we can prove non-negativity by using the definition of matrix mulitplication: 

\begin{equation}
\begin{aligned}
(A^TA)_{ij} = \sum\limits_{k}(A^T)_{ik}A_{kj} = \sum\limits_{k}A_{ki}A_{kj}
\end{alighed}
\end{equation}

\medskip
\medskip
\medskip
\noindent
And:

\begin{equation}
\begin{aligned}
\mathrm{Tr}(A^TA) = \sum\limits_{k}(A^TA)_{ii}= \sum\limits_{}\sum\limits_{k}(A_{ki})^2
\end{alighed}
\end{equation}

\end{document}
